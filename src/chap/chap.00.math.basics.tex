\chapter{数学基础}\label{chap:math-basics}
\addtocontents{los}{\protect\addvspace{10pt}}

\begin{intro}
物理的学习离不开数学,但学习时间的有限使得很多同学纠结数学应该学到多深。在普通物理的范畴内,作者在这里提出一些小小的建议。

数学知识分两类。

第一类是需要完全理解和掌握的,比如数列、导数和积分、微分方程……它们在题目中出现的概率很高,当我们遇到时要能从容应对。

第二类是有助于物理理解,但不要求完全掌握的,比如矢量分析与场论、张量代数……它们有助于加深你对物理情景的理解,但在普通物理的题目中出现的概率较低,这种知识需要学习,但学习的深度要视自身情况而定。
\end{intro}

\section{一元函数微积分}\label{sec:single-variable-calculus}

\subsection{微分}\label{subsec:differential}

一元函数可记为
\begin{equation}
    y = y(x)\ \text{或}\ y = f(x),
\end{equation}

在它的连续区间内,如图 \ref{fig:function-increment}

\input{figure/chap.00/fig.function-increment.tex}

所示,自变量由 $x$ 变到 $x + \Delta x$,相应的,$y$ 由 $y(x)$ 变到 $y(x + \Delta x)$,则函数的增量定义为
\begin{equation}
    \Delta y = y(x + \Delta x) - y(x).
\end{equation}

自变量 $ \Delta x \to 0$ 时,称为自变量微分,记为 $ \d x$,$ \d x$ 是无穷小量,不是零。
但因为它是无穷小量,它在与有限量的运算中,在一些情况下可以视为 0 (但不是在所有情况下都可以视为 0)。
在连续区间内,自变量增量 $\Delta x$ 趋近于微分 $\d x$,函数增量 $\Delta y \to \d y$,称为函数微分,记为 $\d y$,它也是无穷小量。
$ \d y $ 与 $ \d x $ 之间的关系为
\begin{equation}
    \d y = y (x + \d x) - y (x).
\end{equation}

\begin{note}{符号说明}{note:symbol-explanation}
    显然,因为 $ y(x) $ 不一定是单调函数,所以 $ \Delta y $ 和 $ \d y $ 可以是正的、负的或零。
\end{note}

\begin{example}{重要近似}{example:important-approximation}
    \begin{equation}
        \sin x = \tan x = x, (x \to 0).
    \end{equation}
    \input{figure/chap.00/fig.important-approximation.tex}
    以 $ O $ 为原点建立直角坐标系,绘出以 $ R $ 为半径的圆弧如图 \ref{fig:important-approximation} 所示,
    其中圆心角 $ \theta $ 对应的直线段 $ AA' $ ,$ BB' $ ,圆弧 $ \wideparen{AB} $ 的长度分别为
\end{example}




